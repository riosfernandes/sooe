\chapter{Servi�o}

Servi�o � uma atividade na qual a construtora est� apta a realizar. Ex.:

\begin{itemize}
	\item Analise granulometrica sem sedimentacao;
	\item Ensaio para determinacao do Indice Suporte California (CBR) - 3 pontos - obtido com energia Proctor Intermediario, atraves de, no minimo, 5 corpos de prova, conforme recomendacao da NBR9895, NBR6457, NBR7182; ou
	\item Alvenaria de tijolo macico (7x10x20)cm, com argamassa de cimento e saibro no traco 1:6, em paredes com vaos ou arestas, de meia vez (0,10m), ate 3m de altura, e medida pela area real.
\end{itemize}

Inclusive alguns servi�os como o uso de engenheiros da contrutora como se fosse uma consultoria.

\section{Recursos}

Recurso � tudo que pode ser usado para desenvolver determinado servi�o, podemos ter servi�os desses tipos:

\begin{itemize}
	\item Humano: Pessoas que desenvolveram a atividade proposta. Ex., engenheiros, estagi�rios, pedreiros etc;
	\item Material: Usado para construir as obras. Ex., cimento, areia, madeira etc;
	\item Equipamento: s�o ferramentas usadas para desenvolver um trabalho. Ex., pincel de tinta, britadeira, bate estaca etc.
\end{itemize}

\subsection{Unidade de medi��o}

Cada recurso possui uma unidade \cite{quilometro, horas, metros etc} para calcular seu custo e/ou utiliza��o, exemplo, o recurso do tipo equipamento chamado britadeira possui seu custo calculado com base na unidade horas.

\subsection{Custo}

Cada unidade do recurso possui um custo vinculado, ou seja, se o recurso do tipo equipamento chamado britadeira que possui a unidade em horas � utilizado por 2 horas o mesmo ter� o custo de 2 













\subsection{Composi��o}

Vinculo Entre Servi�o e Recurso...

Para realizar determinado servi�o ser�o necess�rios usar recursos estipulados. Ex., para fazer o recurso ``Cobertura em telhas onduladas, sem amianto, com espessura de 4mm, fixadas por pregos, inclusive vedacao, exclusive o madeiramento, Vogatex ou similar. Fornecimento e colocacao.'' ser� necess�rio usar os seguintes recursos:



\begin{itemize}
	\item 3% incidente sobre mao de obra direta com Encargos Sociais para cobrir despesas de EPI e ferramentas;
	\item Conjunto de vedacao para telha ondulada (arruela galvanizada com borracha)
	\item Prego com cabeca, de (18x30);
	\item Telha ondulada sem amianto, com espessura de 4mm, medindo: (2,44x0,50)m, Vogatex ou similar;
	\item Carpinteiro - forma de concreto;
	\item Servente Tributos sobre o faturamento (7.56%);
\end{itemize}















\section{Unidade de medi��o}

Todo sevi�o possui uma unidade de medi��o (quilometro, horas, metros etc), justamente para determinar o pre�o do servi�o. Ex., a ``Alvenaria de tijolo macico'', mostrada acima � cobrada por metro quadrado (m2), ou seja, para cada m2 � utilizado todos os recursos alocados (nas suas devidas propor��es j� pr�-determinadas) e tem um custo j� estipulado.

\section{Outros servi�os}