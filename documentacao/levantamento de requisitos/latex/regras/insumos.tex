\chapter{Insumos}

Insumo é tudo que representa unidade e que compõe as atividades (serviços) de um projeto.

\section{Tipo de Insumo}

O insumo é subdividido em: 

\begin{itemize}
	\item Materiais;
	\item Mão de Obra; e
	\item Equipamento.
\end{itemize}

\section{Unidade de Insumo}

Os insumos são medidos por unidades:

\begin{itemize}
	\item Kilograma (kg);
	\item Litro (l);
	\item Hora/Homem (h);
	\item Metro cúbico (M3);
	\item Metro quadrado (M2); e
	\item Outros.
\end{itemize}

O custo do insumo é dado por unidade, como demonstra a tabela \ref{tab:composicao}. Os valores são representados para uma única unidade de cada insumo, sendo que uma atividade poderá consumir \emph{n} quantidades de insumos.

Um insumo pode apresentar diversos tributos que incidem sobre o mesmo, como por exemplo um insumo de mão de obra poderá apresentar tributos de INSS, ICMS, ISS, etc. Todos os tributos são medidos por unidade de percentual (\%). Esses insumos também podem estar presentes nos serviços. Veremos isso no capítulo dos serviços. A tabela \ref{tab:insumos_exemplo1} demonstra a incidência de tributos sobre um recurso.

\begin{table}[h]
	\centering
	\begin{tabular}{|l|l|c|r|}
	\hline
	Código&			Descrição&						Unidade&		Valor 	\\ \hline
	1.0.0.0&		Ajudante geral&					h&				10,00	\\ \hline
	\end{tabular}
	\begin{tabular}{|l|l|c|r|}
	\hline
	Código&			Descrição&						Unidade&		Valor 	\\ \hline
	TRI0001&		INSS&							\%&				15,00	\\ \hline
	TRI0002&		ICMS&							\%&				3,00	\\ \hline
	TRI0003&		ISS&							\%&				1,50	\\ \hline
	\end{tabular}
	\caption{Incidência de tributos sobre um recurso}
	\label{tab:insumos_exemplo1}
\end{table}

Um exemplo para a atividade \emph{Alv. Bl. Concreto 9x19x39} para construção de uma parede de 8m x 3m (24 m2). Assim, de acordo com a tabela de composição do serviço \ref{tab:composicao}, o serviço \emph{Alv. Bl. Concreto 9x19x39 vedação} tem um custo de 39,91 por m2. Para a execução da parede utilizando a atividade acima, seriam necessárias 24 unidades, totalizando seu custo em 957,84.