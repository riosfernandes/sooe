\chapter{Orçamento}

O sistema deverá permitir:

\begin{itemize}
	\item Cadastrar orçamento;
	\item Definir a situação do orçamento:
	\begin{enumerate}
		\item estimativa;
		\item venda(inicial); e
		\item execução.
	\end{enumerate}
	\item Associar serviços ao orçamento.
\end{itemize}

Para cadastrar um novo orçamento é necessário que o projeto esteja em situação ``em aberto''.

Um orçamento de ``venda'' (comumente nomeado de ``inicial''), é obtido através de um orçamento de estimativa. 

Um orçamento de ``execução'' é obtido através de um orçamento de venda.

Quando um orçamento de execução for definido, a fase do projeto deverá ser alterada para ``fase de execução''.

Um projeto de ``venda'' normalmente pode sofrer alterações e/ou aditivos (recursos ou serviços não previstos no orçamento de venda). Essas alterações devem ser registradas no orçamento de ``execução''. Já alterações quando de vontade do cliente devem ser feitas no orçamento de ``venda'' e \emph{\textbf{refletidas}} no orçamento de ``execução''.