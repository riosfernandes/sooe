\chapter{Orçamento}

O sistema deverá permitir:

\begin{itemize}
	\item Cadastrar orçamento;
	\item Definir a situação do orçamento:
	\begin{enumerate}
		\item estimativa;
		\item venda(inicial); e
		\item execução.
	\end{enumerate}
	\item Associar serviços ao orçamento.
\end{itemize}

Para cadastrar um novo orçamento é necessário que o projeto esteja em situação ``em aberto''.

Quando um orçamento tiver sua situação alterada para ``venda'', o projeto deverá ter sua fase alterada para ``fase de aprovação''.

Quando um orçamento de execução for definido, a fase do projeto deverá ser alterada para ``fase de execução''.

Os serviços dentro do orçamento devem ser filtrados por de ``custo direto'', ``custo indireto'' e ``custo administrativo''. Essa atribuição pode ser definida pelo responsável, podendo futuramente definir outros tipos de custos dentro de seu orçamento. Sendo assim, a apresentação dos serviços em um orçamento deve ser agrupada com base no tipo de custo definido.

Um projeto de ``venda'' normalmente pode sofrer alterações e/ou aditivos (recursos ou serviços não previstos no orçamento de venda). Essas alterações devem ser registradas no orçamento de ``execução''. Já alterações quando de vontade do cliente devem ser feitas no orçamento de ``venda'' e \emph{\textbf{refletidas}} no orçamento de ``execução''.