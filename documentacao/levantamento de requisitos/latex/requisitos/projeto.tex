\chapter{Projeto}
O projeto de orçamento deve possuir:

\begin{itemize}
	\item Um projeto pai\footnote{não obrigatório};
	\item Descrição do projeto;
	\item Cliente;
	\item Endereço (endereço físico do cliente);
	\item Orçamentos (orçamento de estimativa, de venda e de execução);
	\item Situação do projeto;
	\item Tipo de projeto (neste caso de orçamento);
	\item Fase do projeto; e
	\item \% de bonificação (lucro com o projeto).
\end{itemize}

O sistema deverá permitir a criação e alteração de um projeto. Para criar um novo projeto o usuário deve informar:
\begin{enumerate}
	\item Projeto pai (se existir);
	\item O cliente ao qual se destinará o projeto;
	\item O endereço do cliente onde o projeto será executado;
	\item A descrição do projeto;
	\item O tipo do projeto; e
	\item O percentual de bonificação (lucro com o projeto).
\end{enumerate}

Ao criar um projeto, o usuário pode utilizar um projeto como base para a formação do novo projeto, bastando localizar e informar o projeto modelo desejado.