\chapter{Serviço}

Um serviço pode ser reutilizado em outros projetos e/ou orçamentos.
O serviço pode ser agrupado em vários serviços.

Um serviço poderá conter:

\begin{itemize}
	\item Outros serviços\footnote{não obrigatório};
	\item Uma categoria;
	\item Um clã;
	\item Uma família;
	\item Um Grupo (que irá agrupar serviços);
	\item Código (SCO);
	\item FGV;
	\item Descrição; e
	\item Um \% de bonificação (lucro/ serviço).
\end{itemize}

Um serviço é composto de recursos e/ou outros serviços. Essa junção de recursos e/ou serviços ao serviço dá-se o nome de composição.

O grupo é utilizado para agrupar serviços, deve possuir:

\begin{itemize}
	\item Outro grupo\footnote{não obrigatório}; e
	\item Descrição;
\end{itemize}

A categoria, o clã e a família são para categorizar os serviços.

Uma categoria pode possuir vários clãns, um clã pode possuir várias famílias.

Uma categoria pode deve possuir:

\begin{itemize}
	\item Descrição; e
	\item Sigla da categoria;
\end{itemize}

Um clã deve possuir:

\begin{itemize}
	\item Descrição; e
	\item Sigla do clã;
\end{itemize}

Uma família deve possuir:

\begin{itemize}
	\item Descrição; e
	\item Sigla da família;
\end{itemize}